\section{Multiple Sclerosis}
This chapter will review multiple sclerosis (MS) as a disorder, given the importance of MS-related TN as an independent TN subtype. MS is a chronic autoimmune demyelination disease of the central nervous system. It is a progressive disease characterized by wide-spread lesion, or plaques, within the brain, spinal cord, and optic nerve. Its pathogenesis is related to the focal inflammation and damage of the central myelin, where the disruptions of the myelin sheath negatively affect signal transduction of the central white-matter, and thereby result in severe disability. MS symptoms can include cognitive deficits, fatigue, depression, motor weakness affecting the upper and lower extremities and neuropathic pain. Currently there is no cure for MS. An estimated 2,300,000 people world wide are affected by MS, and 50\% of the patient will require motor assistance 15 years after disease onset \cite{Goldenberg2012}. The course of MS can be highly variable due to the distinct plaque distribution pattern in each individual, and therefor MS is difficult to diagnose and treat.

MS is divided into four major clinical groups: 1) Relapsing-remitting (RRMS), 2) Secondary progressive (SPMS), 3) Primary-progressive (PPMS), and 4) Progressive-relapsing (PRMS). RRMS is the most common form of MS, affecting about 85\% of the people with MS. It is characterized by sudden relapse (major flare-up) of symptoms, followed by periods of remission where symptoms disappear. SPMS may develop in some RRMS patients, where the symptoms worsen or level off. PPMS occur in 10\% of the MS population, where there are no remission periods, and symptoms will steadily worsen, with occasional plateaus. PRMS is the rarest form of MS, with less than 5\% of the MS population affected. It is progressive in the beginning, with periods of relapse, it has no remission periods. 

Positive MS diagnosis is satisfied by A) At least two different lesions are present in the CNS white matter (space dissemination); B) At least two different episodes must occur in the disease course (time dissemination); and C) Chronic inflammation of the CNS. One or more of these criteria result in a general diagnosis of MS. Both space and time dissemination criteria are confirmed with clinical MRI. Detailed disability ratings are confirmed by the Expanded Disability Status Scale (EDSS)\cite{Kurtzke1983}, which tracks the progression of MS symptoms in an individual.

MS is defined by the existence of focal demyelination in the CNS that manifests as lesions or plaques. Ultrastructural lesion analysis shows demyelination in the form of myelin debris, macrophages, lymphocytes and Creutzfeldt-Peters cells. Partial remyelination by oligodendrocytes can be present, as well as signs of axonal swelling and injury. In tissue samples of longstanding MS there is no presence of acute demyelination, and heavy loss of axons and myelin could be observed \cite{Filippi2012}. 

\subsection{MR Imaging}

MR imaging is an important component of MS diagnosis, which permits more confident, and early detection of MS when combined with symptom criteria. The most important of these is the McDonald Criteria \cite{Polman2011}, and more recently the MAGNIMS guidelines \cite{Filippi2016}. 

The MS lesions or plaques are characterized as focal hyperintensities in T2-weighted or proton-density images. Chronic and acute lesions could be distinguished by the presence of blood-brain-barrier (BBB) breakdown with the use of MR contrast agents such as gadolinium. Focal MS lesions commonly appear as round hyperintensities in MR images ranging from a few milimeters to centimeter in size. They are also found densely dispersed in paraventricular white matter. The presence of MR hyperintensities is not unique to MS, therefore the location, shape and time dissemination must be considered for the diagnosis. The MAGNIMS guidelines for space dissemination recommends the identification of T2 lesions in at least two of five locations (cortical/juxtacortical, periventricular, infratentorial, optic nerve, and spinal cord) \cite{Filippi2016}. Paraventricular region requires at least three or more lesions to be confirmed. Time dissemination recommends the presence of at least two new lesions in a follow-up MRI when compared to a baseline, or the simultaneous presence of asymptomatic lesions. 

Diffusion MRI (dMRI) is found to be sensitive to tissue changes such as edema, axonal injury and demyelination, and is becoming an important MS tissue quantification strategy. A number of diffusion metrics (Mean diffusivity (MD), radial diffusivity (RD), and axial diffusivity (AD)) have become important in MS tissue quantification and analysis.  They are particularly useful when study normal-appearing white matter. Normal-appearing white matter (NAWM) is microscopically abnormal white matter tissue that is visually indistinguishable from normal white matter in T2 MRI images. They are generally defined to be at least 1 cm away from a lesions. NAWM is found to include microscopic demyelination, gliosis, small round cell infiltration, macrophages and microglial activations. Only about 28 \% of NAWM is found to be normal, and there is a 12-42 \% reduction in axonal density in these tissues. DMRI study of NAWM have found decreased fractional anisotropy and increased mean diffusivity \cite{Ciccarelli2003d,DeGroot2013}.  

\subsection{Pain and MS}
Pain is a common co-morbidity of MS, with up to 50-85\% prevalence \cite{Osterberg2005}. Common pain symptoms include acute pain such as optic neuritis and painful tonic seizure, as well as chronic pain such as trigeminal neuralgia. 91\% of the patients report pain at the time of evaluation, and most experience daily pain. Both central and peripheral pain are reports in people of MS. Central pain in MS is reported to occur 27-58\% of the subjects \cite{OConnor2008}. 

The most common pain in MS includes extremity pain and TN (MS-TN), which is also termed symptomatic TN. MS-TN has a prevalence 2\% in the MS, which is 20 times higher than in a general population \cite{Cruccu2009}. The clinical presentation of the pain is similar to classic TN. MS-TN has a prevalence rate of about 2-4\%, and in some studies, there are reports of prevalence rate as high as 6\%. Unilateral MS-TN is the most common pain type. However, there are up to 30\% occurrences of bilateral pain. MS-TN pathophysiology is believed to be different from classic TN, as it is often accompanied by the presentation of pontine MS lesions \cite{Cruccu2009}. The treatment of MS-TN via ablative interventions is also less effective than in classic TN \cite{Nurmikko2009}. The exact pathophysiological differences between MS-TN and classic TN, however, are not known. While it shares the feature of vascular compression with classic TN, its relevance to MS-TN pain has not been precisely determined \cite{Nurmikko2009}.