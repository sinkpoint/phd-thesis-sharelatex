\textbf{Introduction:}  Imaging of trigeminal neuralgia (TN) has demonstrated key DTI based diffusivity alterations in the trigeminal nerve but imaging has primarily focused on the peripheral nerve segment, since technical limitations have  prevented us from assessing the trigeminal nerve and its connections in the central nervous system. We apply fully-automated group white-matter tractography to image the trigeminal sensory pathway and machine learning classification using a Gaussian Process (GP) classifier to pinpoint key white-matter diffusivity changes that maximally differentiate TN subjects from healthy controls. 

\textbf{Methods:} We used SAGIT based group merged tractography to study 36 sex-matched TN subjects (right-sided pain) and 36 controls, examining the following trigeminal related white matter pathways: trigeminal nerve (CN V), pontine decussation (TPT), and thalamocortical fibres (S1). GP classifiers were trained by scrolling a moving window over CN V, TPT, and S1 tractography centroids. Fractional anisotropy (FA), generalized FA (GFA), radial diffusivity (RD), axial diffusivity (AD), and mean diffusivity (MD) metrics were evaluated for both groups, analyzing TN vs. control groups and affected vs. unaffected sides. Classifiers that performed at greater-or-equal-to 70\% accuracy were included.

\textbf{Results:} GP classifier consistently demonstrated bilateral trigeminal changes, differentiating them from controls with an accuracy of 80\%, even though the clinical expression of pain was strictly unilateral. Affected and unaffected sides could also be differentiated from each other with an accuracy of 75\%. Assessment of the trigeminal subregions demonstrated that the segments in affected CN V differ from controls from the region of the REZ and distally. Bilateral TPT could be distinguished from controls with at least 85\% accuracy while TPT left-right classification achieved 98\% accuracy. Bilateral S1 could be differentiated from controls, where the affected S1 RD classifier achieved an accuracy of 87\%.

\textbf{Conclusions:} This is the first study in TN that combines group-wise merged tractography, machine learning classification using GP, and analysis of the complete trigeminal pathways from the the trigeminal peripheral fibres to S1 cortex. This analysis demonstrates that TN is characterized by bilateral changes throughout the trigeminal pathway compared with healthy controls, as well as changes between clinically affected and unaffected sides. The combination of group tractography and machine learning have proven to be a powerful approach for the study of white matter diffusivities.