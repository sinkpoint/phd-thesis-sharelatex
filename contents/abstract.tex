\begin{abstract}
% \addcontentsline{toc}{chapter}{Abstract}
Idiopathic (Classic) Trigeminal Neuralgia (TN) is a facial neuropathic pain syndrome characterized by paroxysmal, shock-like pain condition affecting one or more of the three trigeminal nerve (CNV) branches. TN is believed to be associated with nerve-vascular compression in the CNV root-entry-zone, but its pathophysiology is still unclear. Single-tensor diffusion tensor neuroimaging (DTI) studies of CNV revealed diffusivity changes in the cistern segment. However, the portion of the nerve within the brainstem remained elusive due to DTI limits. Diffusion imaging of TN is also error-prone due to manual data processing and analysis. 

This thesis aims to: 1) develop a fully-automated software framework to analyze diffusion tractography to reduce error and increase speed of experiments. 2) to apply the method developed to the analysis of the trigeminal system in a more substantial patient group, and apply state-of-the-art methods to advance the study of TN further. 

The specific aims are: a) establish the feasibility of multi-tensor tractography of brainstem CNV; b) establish the best way to minimize DWI to T1 co-registration error across multiple subjects; c) Create the software framework to generate and quantify tractography at the group level; d) apply the methodology to the reconstruction and quantification of the trigeminal sensory pathway. 
Towards these goals, in Study I, we establish the feasibility of applying multi-tensor tractography to delineate the full course of CNV and demonstrate that TN is uniquely identified by disruptions in the cistern/REZ, while MS-TN by disruptions in the brainstem course of the nerve. In Study II, we determine that the best T1-DWI co-registration scalar is the Mean DWI image. In Study III, we present the Selective Automated Group Integrated Tractography (SAGIT) processing pipeline framework. Finally, in Study IV, we deploy end-to-end machine-learning TN classification to automatically discover diffusivity disruptions in the cistern/REZ CNV, the trigeminopontothalamic decussaion, and thalamocortico S1 pathway. 

In sum, this thesis presents a detailed road-map of the development and application of end-to-end diffusion tractography machine learning classification. The application to TN revealed specific diffusivity changes in trigeminal CN V, pontine, and S1 white matter pathways, and pin-points the locations of the diffusivity disruptions at millimetre level.
\end{abstract}