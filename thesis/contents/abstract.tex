Idiopathic (Classic) Trigeminal Neuralgia (TN) is a facial neuropathic pain syndrome characterized by paroxysmal, shock-like pain condition affecting one or more of the three trigeminal nerve (CNV) branches. TN is believed to be associated with nerve-vascular compression in the CNV root-entry-zone, but its pathophysiology is still unclear. Single-tensor diffusion tensor neuroimaging (DTI) studies of TN revealed diffusivity changes in the cistern CNV. However, the portion of the nerve within the brainstem remained elusive due to the limits of DTI, and the lack of reliable software to researchers. Thereby the study of TN using diffusion imaging is time-consuming and error-prone due to the complete manual involvement in all steps of data process, placement of region-of-interest (ROI), and tractography parameter tuning, and data analysis. 
There are two aims of this thesis: 1) to develop a fully-automated software framework to collate and analyze diffusion tractography to reduce human error and increase iteration speed of experiments. 2) to apply the method developed to the analysis of TN in a more substantial patient group, and apply state-of-the-art diffusion imaging methods to advance the study of TN further. 
The specific aims are: a) establish the feasibility of high angular resolution diffusion tractography of brainstem CNV; b) establish the best way to minimize DWI to T1 co-registration error allow non-linear registration of tractography across multiple subjects; c) Create the software framework to generate and quantify tractography at the group level; d) apply the methodology to the reconstruction and quantification of the trigeminal sensory pathway. 
Towards these goals, in Study I, we establish the feasibility of applying spherical deconvolution tractography to delineate the full course of CNV from the trigeminal nucleus and demonstrate that TN is uniquely identified by disruptions in the cistern/REZ, while MS-TN can be distinguished by disruptions in the brainstem course of the nerve. In Study II, we discover the best T1-DWI co-registration scalar is the Mean DWI image, and that the registration can potentially be improved with a novel spherical harmonic parameter scalar imaged called Anisotropic Power image. In Study III, we present the Selective Automated Group Integrated Tractography (SAGIT) software framework. Finally, in Study IV, we present an end-to-end machine-learning TN classification study that auto-discovers diffusivity disruptions in the cistern/REZ CNV and thalamocortico S1 pathway. 
In sum, this thesis presents a detailed road-map of both the development and application of a novel end-to-end diffusion tractography machine learning classification. The application of this system to TN revealed specific diffusivity changes in deeper white matter structures, and pin-points the exact location of the diffusivity disruption at millimetre level. 