\chapter{Aims/Hypotheses}

The general aim of this thesis is to quantify the specific group-wise diffusivity changes along the trigeminal sensory white matter pathways, in order to identify specific diffusivity pattern that differentiates between TN and non-TN. Towards this aim, a generalizable group diffusion imaging tractography software framework must be developed to overcome the technical barrier of large-scale group tractography. The roadmap and rational for the specific studies are described below.

\section{Study I: Diffusivity signatures characterize trigeminal neuralgia associated with multiple sclerosis}
MS-TN, also known as Symptomatic TN, is thought to relate to the brainstem MS lesions. However the exact anatomical relationship between the lesions and trigeminal nerve within the brainstem has been difficult to quantify, due to the limited contrast and resolution in standard MRI, and the variability of MS lesion across individuals. Due to these factors, image-registration-based techniques such as voxel-based morphometry becomes unreliable. Previous single-tensor tractography studies of CN V has been limited to the cistern nerve segment. In this study, we have the opportunity to study sub-segments of CN V in TN and MS-TN by using multi-tensor tractography to delineate brainstem CN V. Thereby establishing the feasibility of multi-tensor tractography in improving anatomical delineation of CN V, as well as its usefulness in the study of TN. 

\subsection{Main Aim} 
Establish the feasibility of diffusion tractography in delineating the brainstem CN V, and contribute towards the study of pathological CN V tissue.

\subsection{Specific Aim}
\begin{itemize}
    \item Use multi-tensor tractography to delineate brainstem CN V, in order to quantify tissue diffusivity measurements in targeted sub-regions.
    \item Differentiate TN and MS-TN based solely on the obtained CN V diffusivity measurements.
\end{itemize}

\subsection{Specific Hypothesis}
\begin{itemize}
    \item TN is known to relate to vascular compression of the cistern REZ, therefore its diffusivity signature will occur in the cistern region.
    \item MS-TN should be related to brainstem MS lesions. Therefore CN V segments close to the lesions will show more significant changes in tissue diffusivity, whereas segments far away from the lesions will not. 
    \item MS is a demeylination disease, therefore MS-TN diffusivity signature must be related to demyelination, such as increase in RD, and decrease in FA. 
\end{itemize}

\section{Study II: Diffusion Weighted Image to T1 Co-registration Without Reverse-blip: Investigation of Best Practices}
The task of establishing inter-subject anatomical correspondences is imperative for extending tractography to multiple subjects. Population studies with inter-subject T1 registration are well-established, where a normalized T1 template is generated from numerous subjects to facilitate data analysis. Similarly, DWI of each subject can be co-registered to their corresponding T1, to extend the T1 template space to tractography space. Therefore the DWI-T1 co-registration accuracy is crucial to the reliability of inter-subject tractography alignment. DWI-T1 co-registration is challenging due to the DWI eddy-current and EPI field distortions. While new DWI sequences with reverse-blips have been shown to reduce non-linear distortions, they have yet to be commonly adopted in clinical practice. While T1 inter-subject registration methodology is well documented, the feasibility, and strategy for DWI-T1 co-registration on sequence without reverse-blips have not been documented, and forms a potential bottleneck on the performance of any DWI-T1 population registration pipeline.

\subsection{Main Aim}
Establish the best strategy to minimize DWI to T1 co-registration error in order to allow multi-modal template space registration across multiple subjects.

\subsection{Specific Aim}
\begin{itemize}
    \item Quantify the degree of improvement in registration between affine-only versus symmetric-diffeomorphic registrations.
    \item Determine which commonly-used DWI scalar image types can provide the best registration result when used as the registration-intermediate.
    \item Determine if anisotropic-power image, an experimental DWI scalar map that closely resemble T1 contrast, can provide even more registration improvements. 
\end{itemize}

\subsection{Specific Hypothesis}
\begin{itemize}
    \item Symmetric-diffeomorphic registration should result in the best accuracy, but with more diminishing asymptotic improvement over affine registration alone.
    \item Averaged DWI should be the best scalar image, as it provides a balanced contrast between white and grey matter.
    \item Anisotropic-power Image will show more substantial improvement over the existing scalar images.
\end{itemize}

\section{Study III: Merged Group Tractography Evaluation with Selective Automated Group Integrated Tractography}
Once the feasibility of multi-tensor tractography to the brainstem trigeminal pathways is established, and DWI-T1 co-registration optimization identified, the software framework for group tractography is constructed. The software aims to automate the tractography process at large-scale, and also aims to reduce the cognitive load of researchers by providing region-of-interest, and tractography parameter management. It also provides the ability to report the tractography output at a group level both visually, as well as quantifying the output by an original scalar score called Normalized Overlap Score (NOS) score. The study demonstrates the software's flexibility by comparing the performance of four tractography algorithms across six anatomical regions in a healthy population. At a scale that would be time-prohibitive with manual methods. The comparisons determine the best tractography methods to use for different types of neuroanatomy. 

\subsection{Main Aim}
Make possible non-linear registration of tractography geometries, and to quantify the geometry registration performance at the group level

\subsection{Specific Aim}
\begin{itemize}
    \item Compare DTI, XST, deterministic constrained-spherical deconvolution (CSD), and probabilistic CSD in 42 subjects. 
    \item Delineate neuropathways that are known to be difficult to delineate: fornix, facial/vestibular-cochlear cranial nerve complex, vagus nerve, rubral-cerebellar decussation, optic radiation, and auditory radiation.
    \item Determine if NOS score perform as expected on real-world data, by comparing with manual rating by neuroanatomy experts.
\end{itemize}

\subsection{Specific Hypothesis}
\begin{itemize}
    \item At a group level, the delineations should show reasonable convergence, and clearly recognizable. 
    \item The group tractography should reveal known limits of DTI, as well as improvements on XST and CSD methods.
    \item Deterministic tractography should preferentially be more suitable for nerve tractography, and probabilistic method better for deep neural pathways such as the optic and auditory radiations.
    \item NOS score should correlate with expert ratings, and show improved consistency.
\end{itemize}

\section{Study IV: Gaussian Process Classification of Trigeminal Neuralgia With Merged Tractography}
With the software framework thoroughly tested, and the best-suited tractography methods determined, the developed methods are applied to the trigeminal sensory pathway. The study aims to examine the diffusivity signature differences in the trigeminal sensory pathway between TN and controls in a larger group. The pathway is divided into three major regions: CN V, trigeminopontothalamal (TPT) segment, and the thalamocortical S1 segment. With the increased volume of data, more detailed along-the-tract analysis can be carried out on each segment, and machine learning is used to auto-determine relevant segments that maximally differentiates TN from controls to minimize human bias. S1 and thalamus activation were consistently found in chronic pain fMRI, and therefore their related pathway should also show similar findings. 

\subsection{Main Aim}
Apply previously established methodology to the reconstruction and quantification of the trigeminal sensory pathway. 

\subsection{Specific Aim}
\begin{itemize}
    \item Delineate and visualize the three segments of the trigeminal sensory pathway: CN V, trigeminopontothalamal segment, and the thalamocortical S1 segment. 
    \item Determine diffusivity signatures that maximally differentiate the affected and unaffected sides of the TN patients.
\end{itemize}

\subsection{Specific Hypothesis}
\begin{itemize}
    \item CN V diffusivity signature should match findings from Study I.
    \item At the level of CN V, the affected cistern segment should show differentiation from the unaffected side.
    \item The affected S1 segment should differ from the unaffected side, due to more frequent nociception.
    \item The affected TPT segment might show more differentiation near the end points, due to possible differences in trigeminal nucleus and thalamus diffusivity. However, its mid-segment diffusivity might be inconclusive, as the diffusivity metrics such as FA and RD might not be adequate in describing the complex crossings at the pontine decussation. 
\end{itemize}
