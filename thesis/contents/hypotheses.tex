\chapter{Aims/Hypotheses}

The general aim of this thesis is quantify the specific group-wise diffusivity changes along the trigeminal sensory white matter pathways in order to identify specific diffusivity pattern that differentiates between TN and non-TN CN V tissue. In achieve this aim, a generalizable group diffusion imaging tractography software framework must be developed to overcome the technical barrier of large scale group tractography in arbitrary populations. A number of confounds on data registration and measure must be resolved in order to gain confidence in the result of this methodology. These include a) establish the feasibility of diffusion tractography of brainstem CN V originating from the trigeminal nucleus; b) establish the best way to minimize DWI to T1 co-registration error in order to allow multi-modal template space registration across multiple subjects; c) Make possible non-linear registration of tractography geometries, and to quantify the geometry registration performance at the group level; d) apply previously established methodology to the reconstruction and quantification of the trigeminal sensory pathway. 

