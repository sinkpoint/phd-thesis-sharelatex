\chapter{Aims/Hypotheses}

The general aim of this thesis is quantify the specific group-wise diffusivity changes along the trigeminal sensory white matter pathways in order to identify specific diffusivity pattern that differentiates between TN and non-TN. In achieve this aim, a generalizable group diffusion imaging tractography software framework must be developed to overcome the technical barrier of large scale group tractography in arbitrary populations. A number of confounds on data registration and measure must be resolved in order to gain confidence in the result of this methodology. The roadmap and rational for the specific studies are described below.

\section{Study I}
Symptomatic TN, or MS-TN is thought to relate to the brainstem MS lesions. However the exact anatomical relationship between the lesions and trigeminal nerve within the brainstem has been difficult to quantify, due to the limited contrast and resolution in standard MRI, and the variability of MS lesion across individuals. Due to these factors, image-registration-based techniques such as voxel-based mophometry becomes unreliable. Previous single-tensor tractography studies of CN V has been limited to the cistern nerve segment. In this study, we have the opportunity to study sub-segments of CN V in TN and MS-TN, with the assistance multi-tensor tractography delineation of the brainstem CN V. Thereby establishing the feasibility of multi-tensor tractography in improving anatoimcal delineation of CN V, as well as its usefulness in the study of TN. 

\paragraph{Main Aim} Establish the feasibility of diffusion tractography in delineating the brainstem CN V, and contribute towards the study of pathological CN V tissue.

\paragraph{Specific Aim}
\begin{itemize}
    \item Use multi-tensor tractography to delineate brainstem CN V, in order to quantify tissue diffusivity measurements in targeted sub-regions.
    \item Differentiate TN and MS-TN based solely on the obtained CN V diffusivity measurements.
\end{itemize}
\paragraph{Specific Hypothesis}

\section{Study II}

\paragraph{Main Aim}
establish the best way to minimize DWI to T1 co-registration error in order to allow multi-modal template space registration across multiple subjects
\paragraph{Specific Aim}
\paragraph{Specific Hypothesis}

\section{Study III}

\paragraph{Main Aim}
Make possible non-linear registration of tractography geometries, and to quantify the geometry registration performance at the group level
\paragraph{Specific Aim}
\paragraph{Specific Hypothesis}

\section{Study IV}
\paragraph{Main Aim}
apply previously established methodology to the reconstruction and quantification of the trigeminal sensory pathway. 

\paragraph{Specific Aim}
\paragraph{Specific Hypothesis}
