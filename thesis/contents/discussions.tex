\graphicspath{{images/discussions/}}

\chapter{General Discussions}

\section{Summary Of Findings}

The contribution of this thesis are as follows:

\begin{itemize}

\item Imaging of the brainstem segment of the CN V \textit{in vivo} using multi-tensor diffusion tractography. Prior to this work, CN V tractography was limited to single-tensor DTI models and trigeminal fibers could not be visualized as they enter into the pons. Clear delineation of the pontine CN V nerve allows reliable measurements of CN V diffusivity, therefore greatly increasing the range of study of trigeminal-related pathology that can be studied with neuroimaging, such as MS-TN. 

\item Differentiation between TN and MS-TN by examining sub-segment diffusivities, and distinguishing between cistern (TN) and brainstem peri-lesional (MS-TN) segments. This pathophysiological distinction between these two entities had not been previously demonstrated using neuroimaging tools, nor had it been correlated with the clinical expression of pain. 

\item Overcoming the major technical barriers to diffusion tractography study of the trigeminal pathway in a large population group, by developing and validating the Selective Automated Group Integrated Tractography (SAGIT) software platform. For the process, we determined the optimal T1-DWI co-registration methodology with clinical DWI sequences.

\item Merged group tractography study of the trigeminal nerve and its primary sensory pathway as a novel use of Gaussian Process machine learning to the study of TN. This study involved spatial defintion of groups tracts in space, as well as measurement of diffusivity metrics from the subregions of trigeminal primary (CN V), secondary (trigeminopontothalamic, TPT), and tertiary (thalamocortical, S1). We demonstrate distinct bilateral changes in trigeminal neuralgia subjects, as well as differences between affected and unaffected sides. 

\end{itemize}

\section{A new approach to the study of Trigeminal Neuralgia}


The introduction of DTI in the study of TN has revolutionized our understanding of this disorder. Moving away from clinical studies alone, DTI is able to derive metrics that provide important information on the microstructure of the nerve which can then be related to specific processes such as changes in myelination or the microstructure of the axons. Thus DTI provides an important objective measure in the study of pain, through a previously unavailable approach. The earliest work on the feasibility of accurate reconstruction of cranial nerve with DTI tractography \cite{Hodaie2010} dates to 2010. This study was able to image CN II, III, and importantly CN V. Similarly, we were able to establish the methods to image the TRG and the cisternal CN V with manual delineation, followed by visualization of the CN V postganglionic GK target, prior and after surgery \cite{Hodaie2012g}. We found alternations in both the CN V reconstruction, as well as FA in the surgical target location, that improved our understanding beyond what is visible with post-surgical gadolinium enhancements. This has now led our team to study whether CN V GK target diffusivity at 6 months post-surgery can predict long-term pain relief in TN \cite{Tohyama2018}. Similarly whether the diffusivity changes are specific enough that they can serve as predictors of outcome \cite{Hung2017}. The tractography study of TN has been directly linked with the study of gray and white matter abnormalities of the TN brain \cite{Desouza2013c}. The value of DTI studies in TN is therefore unique, and creates a strong context for the work done in this thesis. 

The findings in Study I (Chapter \ref{section:study1}), point to the clear role of DTI in distinguishing the localized, focal changes in diffusivity that characterize TN and MS-TN. While TN is associated with compression of the REZ and possible associated focal central demyelination \cite{Devor2002a,Peker2006}, MS is associated with inflammation and myelin changes resulting in CNS plaques. While both diseases involve the same clinical expression of pain, the pathophysiology of pain differs, but this had not been shown directly using DTI based imaging techniques. From a technical perspective, the identification of the CN V nerve inside the brainstem from MR images had been previously highly inconsistent, and the precision of multiple ROI placements could not be maintained. Multi-tensor tractography was instrumental in the feasibility of the method. The study opened the possibility that multi-tensor tractograpy aided analysis added a measure of objective assessment in examining pain conditions, and may facilitate similar studies that provide insight into the fundamental changes that occur at the level of white matter. 

As a result of the methodology developed in Study I, a recent publication details the investigation of whether presurgical CN V segment diffusivity values can predict response to TN surgery \cite{Hung2017}. The important study found that lowered cistern MD and AD pre-surgery was correlated with positive surgical outcome, whereas lowered REZ FA and increased pontine AD was correlated with the lack of response from surgery. This provided further evidence of the role of cisternal CN V diffusivities in the study of TN. The fact that the novel finding of pontine diffusivity in classic TN favours a non-responder outcome points to the role of CNS alterations in classic TN.

Study IV (Chapter \ref{section:study4}), applied group tractography to the study of the trigeminal sensory pathway, and highlighted the observation that the unaffected CN V diffusivity in TN differs from that of controls. There have been other studies suggesting bilateral abnormalities in TN. A group level diffusivity study of TN was conducted by DeSouza et al. in 2013 \cite{Desouza2013} presented similar findings. DeSouza manually defined ROI in the trigeminal REZ, as well as performing full-brain Tract-Based Spatial Statistics. Bilateral differences in the REZ between TN patients and controls were found in MD, RD, and AD. Wide-spread white matter differences were also found, especially in the corpus callosum, cingulum, and longitudinal fasciculus. DeSouza had also found increased cortical and grey-matter volume changes in TN patients \cite{Desouza2013c}, including increase in contralateral S1 and thalamus volumes.
Using SAGIT in study IV, the ROIs were projected to each individual subject using an automated pipeline. The GP classifier produced results by exhaustively examine moving windows at all position and sizes, and the results are auto-discovered by the algorithm, in contrast to manual-placement of the measurement ROI in previous studies. There is therefore consistency in CN V findings in the location and diffusivity across both manual and automated methods.

The observation of bilateral differences has important implications in TN pain. From previous studies, it is clear that altered diffusivities in the cisternal and pontine segments of CN V are correlated with TN pain in the affected side. However there are now a number of studies that point to bilateral REZ/pontine CN V diffusivities being altered in unilateral TN, although each point to significant differences between affected and unaffected sides as well. The exact nature of bilateral diffusivity alterations is not yet clear. Perhaps the earliest indication that TN can affect the nerves bilaterally is a clinical paper by the Burchiel group, where they point to a surprisingly large number of patients with unilateral TN that have evidence of bilateral neurovascular compression, in contrast with healthy controls \cite{Miller2009}. Moreover descending modulation from PAG can become maladaptive or ineffective in chronic pain, result in central changes in the trigeminal nucleus and its associated pathways.

The unaffected side was previously presumed to be similar to normal tissue due to the absence of pain. In many studies it has also been used as a basis of comparison. With this additional insight, it becomes important to not interpret results of affected-to-unaffected comparisons in terms of differences between pathological and healthy tissue. Additionally, the logical next question is to determine that, if the unaffected nerve differs from controls, at what point does the clinical expression of pain start. Investigations into bilateral TN pain, and vascular compression in the absence of pain, may reveal additional insight into the relationship between white matter changes and TN pain in general. The newly developed SAGIT and machine learning methods can additionally be leveraged to conduct large cohort studies, that allow investigation into the brain regions identified by DeSouza et al, as well as examining important areas in the modulation of pain, such as the insula and its relationship to TN. 

Finally, with the methods developed throughout this thesis, future studies are no longer restricted by manual ROI placements for each subject, and tract subregions can be studied in detail. This opens the avenue to the study of associated pathways of TN pain, and allow the development of more sophisticated prediction models to further the understanding, and treatment of TN. 

\section{Tractography of the trigeminal nerve and brainstem fibers}

An important motivation for the use of tractography to study the trigeminal nerve was the ability to overcome the significant limitations of current conventional MR imaging in the visualization of the brainstem trigeminal nerve and the area of the trigeminal nucleus. Prior to the use of tractography, the fibers of the trigeminal nerve and connected pathways could only be inferred from anatomical studies or postmortem dissections. 

The ability to resolve the pontine trigeminal nerve into the trigeminal nucleus signified that for the first time we could use software methods to segment the pontine trigeminal anatomy in a reliable manner. Prior to this, the ROI identification was done visually, and sometimes with simple heuristics methods based on distance from surrounding landmarks. Such manual methods were highly subjective and became problematic with the presence of pathology, where distortions to the nerve could not be always assessed. Improvements in tractography of the brainstem trigeminal nerve allowed the shift towards more algorithmic driven and automated segmentation and ROI placement.

Diffusion tensor imaging revolutionized how we examine brain white matter fibers, yet resolving cross-fibers of the cerebral white matter remains a limitation of diffusion tractography. The brain white matter consists of complex crossing fiber crossings, and the inability for DTI to reliably image these structure therefore poses a fundamental challenge for neuroimaging.
Techniques to improve visualization of white matter fibers consist of probabilistic tractography, which uses tissue priors to sample a large number of directions at each step stochastically, and combines the final result as a volumetric probability map. This kind of image-based probabilistic tractography has proved useful for whole-brain white matter studies. However, it is constrained to the resolution of the diffusion acquisition, which can be insufficient for localized and small diameter pathways. Moreover, the algorithm depends on pre-generated tissue-priors, which excludes cranial nerves. 
Deterministic algorithms such as diffusion spectral imaging (DSI), and the related multi-shell acquisition strategies attempt to model the fiber orientation distribution by dramatically increasing the number of gradient directions (256--512 directions where required) in the diffusion acquisition. The acquisition time of these sequences can be impractical for the clinical environment. Reduction in acquisition time for these algorithms is currently an active area of research.
The family of spherical harmonics (SH) based methods, which include Q-ball and constrained spherical deconvolution (CSD), are a significant development. SH-based methods model the fiber orientation distribution as a decomposition of SH basis functions. This means that these methods are more robust to noise, and therefore require less number of gradient acquisitions. CSD proved to be particularly robust to noise, and can be deployed on acquisitions with only 55 directions. Another family of tractography (XST) algorithm constrained its model assumption to two fiber crossings, also proved to provide good results with 50 gradient directions.

These algorithmic developments brought imaging of crossing-fibers using tractography to the realm of possibility. We explored the feasibility of imaging the pontine trigeminal nerve with CSD and XST. The trigeminal nerve is an ideal imaging target, both due to our immediate interest in TN, and also since the trigeminal nerve has a number of unique properties as a tractography target: 1) Its immediate morphology can be easily assessed in conventional MR images. 2) it courses through simple (cistern) and complex (pons) fiber environments; 3) it is a prominent cranial nerve, and therefore a good benchmark for imaging small anatomy.

As part of our focusing on ??? and for the purposes of imaging the trigeminal pathways, where a significant number of crossing fibers exist, we divided the trigeminal nerve into 30 discrete cross-sections along its path, and observed detailed diffusivity changes as the nerve transitioned from pontine tissue to the cistern. The number of divisions chosen was arbitrary, with consideration that the number 30 could easily be divided into 3 sub-segments to represent the beginning, middle, and end of an elongated structure. In theory the number of divisions could be infinitely large, given that a centroid curve was identified. Determining a common centroid curve from many subjects was computationally non-trivial, especially when the streamline lengths are variable. White matter pathways are highly complex, with possible multi-directional forks, bends, and convergent pathways across a wide range of scales. A generalized algorithm that can reliably calculate a centroid for these structure is of great value and yet still a topic of research. 
Merged tractography  simplified this task by allowing a single curve to be defined manually in the template tractography space. This greatly simplied the task, since centroids no longer need to be defined repeatedly for each subject. Manually curve definition by experts can leverage known techniques in computer graphics to produce centroid for very complex structures, thereby extending SAGIT for obtaining along-the-track measure for other highly variable streamline bundles.

The segmentation of the subregions of the trigeminal pathway has allowed for the possibility of measuring the respective diffusivities. Moreover, this method can extend to any scalar images that can be registered to the DWI space. These may include specific tissue-metrics such as myelin and G-ratio images. This methodology is therefore highly applicable and not specific to the study of TN.


\section{Neuroimaging of deep white matter}

The findings in S1 in Study IV (Chapter \ref{section:study4}) showed TN diffusivity differences in regions surrounding the thalamus, as well as in the mid-segments of the S1 pathways. These findings were in agreement with pain literature, where thalamus and S1/S2 activations are often reported as part of the pain neuromatrix. The TPT pathway resulted in differences near the TGN and the thalamus, areas that are expected to be involved in TN pain. It would be of great interest to re-examine the MS-TN population in these pathways, in order to determine any relationship between deeper white-matter MS-lesion, MS-TN, and non-TN pain in the people with MS.

We have shown in Study III (Chapter \ref{section:study3}) that different tractography algorithms are suited to delineate different neuropathways when comparing merged reconstructions in 42 subjects. Maier-Hein et al came to similar conclusions after examining 96 reconstructions from a single set of simulated data \cite{Maier-Hein2017}. As tractography is applied to the investigation of white matter elsewhere in the brain, care must be taken to ensure  parameters are targeted to the anatomy of study, as more consistent delineation leads to more robust tissue measurements.

Quantification of tissue measures from tractography, as demonstrated in Study IV, implies the use of points in tract models as a sampling method for the scalar values in DWI space. We demonstrated in Study III, that different tractography algorithms exhibit different geometric point distributions. In other words, given two tracts generated from identical ROI definitions, with similar cross-sectional area, but of different algorithm, the underlying spacial distribution of the points are different. It can be argued that we average the point-metrics during along-the-tract analysis, which may be considered a form of smoothing/down-sampling, and therefore the spatial distribution of the tractography algorithm might not be important. However, until the exact influence of spatial distribution have on the measured distribution of diffusivity metrics can be understood, comparing between similar tracts from two different tractography algorithms should be avoided.

\section{Machine Learning}


The availability of the SAGIT, full automation of tractography processing enabled a dramatic increase in the volume of data available for statistical analysis. This lead to some challenges in the scalability of the analytical process, but also opportunities in leveraging more expressive machine learning approaches. The need for data to improve diagnosis and treat of disease, particularly in the emergence of artificial intelligence in medical imaging is increasingly been recognized. Data is now the foundation from which new techniques and discoveries are possible. Initiatives such as the UK Biobank, Allen Brain Institute, and Ontario Brain Institute all collect and provide their neuroimaging data to researchers. With this deluge of imaging data, the complexity of analysis that demand of researchers proportionally increases. The need to include large amount of variables, patient attributes, study measures, time sequences, can now provide valuable insight that is beyond the mean of traditional statistics. In this thesis we developed an machine learning approach that is scalable, and one that can quantify uncertainty as demanded by the medical need. The data processing and machine learning is end-to-end, meaning that the system can be practically deployed in a clinical setting.

We applied machine learning to the analysis of TN in Study IV. The analysis of the trigeminal nerve in TN versus controls agreed with our earlier findings, and therefore validates the study's GP classification method. We performed traditional multiple comparison statistical analysis for our model, but found the result hard to interpret. This was largely due to difficulties in determining the right p-threshold, where the choice of p-value can result in profoundly different statistics. We therefore decided that frequentist statistics was not suitable to our dataset, as we are burdened with too great amount of data, which may not fit with frequentist assumptions. 
The GP classifier was chosen over more popular SVM classifiers, for the following reasons: 1) the obtained metrics show high location dependency. 2) SVM results are hard to interpret, whereas GP is a bayesian method, and therefore it is much more intuitive. More specifically, the streamline measurements resemble time-series data, in that it exhibits positional dependency between measurements. Unlike a time-series however, the streamline measurements represent physical locations along a pathway, therefore they exhibit bi-directional dependency, as opposed to uni-directionally dependent in time. Posing the study as a supervised machine learning problem, the feature vector would be a N dimension vector representing N measures at each division, and a classification label. SVM would assume the features to be points in a N-dimensional space, and each measurement would assume independence. This makes the result of the SVM output difficult to interpret. The GP classifier models the feature vector as a latent GP model, before squashing it to produce a logistic output \cite{rasmussen2006gaussian}. Therefore the GP model is able to take the positional dependency into account. 
Changing the GP classification model into a regression one is very simple. This importance is illustrated thus: Study IV asked the question of at which measurement positions can we maximally discriminate TN from controls at a group level, but it is also possible to pose a prognosis question: Given N measurements along a single patient's CN V, what is the probability that is resembles a pathological population as opposed to a healthy population? It is possible to answer these questions trivially with the GP model using regression. The full implication is outside of the scope of this thesis, but this work will serve as a strong baseline for the applications in this direction. 




\section{Novel Technical developments}

\subsection{Delineating the brainstem fibers of the trigeminal nerve}

For successful tractography of the CN V, both DWI sequence and the algorithm choice are paramount. Given a DWI sequence, cisternal CN V could be readily reconstructed with Gaussian tensors. Unlike major white matter bundles, cranial nerves diameters are in the millimetre range, and therefore voxel sizes greater than 2 mm may result in partial volume averaging. Achieving 2mm isovoxel on a 3T scanner requires 32 channel head coil. Standard clinical 3T MRI scanners with 8 channels head coils can only achieve 2.6 mm isovoxel resolution before acquisition time and noise becomes insurmountable. For this reason, the acquisition in-plane resolution can be tuned to 1x1 mm, and slice thickness at 3mm. While the acquisition is not ideal, it sacrifices resolution only in the z-axis, however the tractography visual fidelity is improved. 
Imaging the brainstem CN V requires algorithms that can reliably resolve tract structures in regions with crossing fibers. A number of methods have been devised, including probabilistic tractography (FSL)\cite{Behrens2007}, which generates image-based results that represent the propagation probability of a seed region. Neurosurgical visualization however requires three dimensional representation of the fibers, and therefore streamline-based tractography visualizations are of broader appeal. In our experience, both XST and sphere harmonic based propagation algorithms can perform adequately to delineate brainstem CN V, as long as there are greater then 50 gradient directions. 

An important technical point, but rarely mentioned in literally, is the tractography seeding pattern. Therefore I will address it here: When tracing the streamlines, an algorithm will need to generate a number of starting points in a defined volume. Common patterns in prepackaged software include in a regularly spaced grid and random sampling. The seeding pattern is important in both the final visual representation of the tracts, and to determine whether key anatomical landmarks can be properly tracked. Another consideration is the reproducibility of the desired anatomy, as randomly seeded bundles cannot be reproduced deterministically. Grid seeding may result in sparse spacing of the tract bundles when viewed from a certain angle, while random seeding can reduce this kind of visual artefact. A good balance might be to have a reproducible sampling scheme that also is robust to different viewing directions, for example a phyllotaxis pattern.

\subsection{Tractography group registration}
Tractography is a powerful methodology that permits the visualization of white matter fibers in three dimensions. Its primary use however has been limited to case studies and small-scale anatomy comparative analysis. The continuous advancements in big data and machine learning demand the adaptation of new neuroimaging methods to the analysis and quantification of larger groups.

Group comparative studies require the establishment of anatomical consensus, such that similar white matter structures between populations can be readily studied. Efforts in MRI medical image registration have paved the way to this end, especially in the fields of diffeomorphic registration. The difficulty in adapting registration to tractography is that tractography data are 3D geometries, and therefore are extremely difficult for analytical methods, as they are computationally untenable. To date there is no direct non-linear 3D deformation methods for tractography. The best efforts to date apply affine transformation for between-tract registrations \cite{Garyfallidis2015}. Our approach is to recast tractography as a brain segmentation method while also make use of its orientation information. Tractography then can be considered a point-based sub-sampling method that is represented in the same coordinate frame as the native DWI space. Any transforms applied to this DWI space is then applicable also to any tractography models that share its coordinate frame. Existing diffeomorphic registration solutions can then be used to non-linearly distort tractography, and circumvent the computational difficulty of geometry based non-linear registration. The advantage of our tractography registration strategy is that any new advances in the field of MR registration can be readily adapted to improve the output of our method. 

Group registration of tractography towards a common template space requires two separate steps: T1---DWI co-registration, and T1---Template registration. Any registration errors in these two steps will become multiplicative in the end result. T1 registration is well studied, however T1---DWI registration has few literature. This demonstrates the need to investigate the best method to minimize registration errors that were compatible with our image data format.

\subsection{Measurements from tracts}
Aggregating along-the-measure statistics is a form of clustering that requires projection of tract vertices onto a structural skeleton, or centroid. Common clustering methods calculate a polyline centroid based on different distancing functions between streamlines. These methods require tractography streamlines to have similar length, and will underperform when tractography structures contain sharp turns. We have explored a number of strategies, and considered adaptation of manifold-based methods, and point-cloud skeletal extraction method from computer graphics literature. These approaches, however, contain assumptions that do not suit our needs. For example, manifolds commonly consider that the latent structure of the data lies on a lower-dimensional surface, while tractography is commonly arranged in complex bundles. Moreover, point-cloud methods deal with point-cloud data obtained from 3D scanners, which situate on a 3D shell, while tractography points are distributed within a volume. White matter anatomy can have intricate crossings, multi-way join and diverging structures. These are difficult for existing 3D analytical methods, and require models with simplified assumptions to reduce the number of unknown parameters. A potentially promising future direction may be to automatically discover these more complex latent structures using deep learning methods.

\section{Limitations}

One limitation across the studies in this thesis was the choice of anisotropic voxels of the DWI sequences. This was limited by the clinical 3-channel GE 3T HDx MR scanner, which places constraints both the acquisition resolution and time. Commonly, DWI voxel resolution on a 32-channel 3T research scanner can obtain up to 2mm isovoxel resolution, on older 8-channel clinical scanners, the maximum isovoxel resolution is limited at about 2.6mm isovoxel. This is insufficient for the study of cranial nerves such as CN V. Our resolution is limited to 1x1x3 mm voxel dimensions, in order to retain in-plane resolution at the sacrifice of slice thickness. Despite this, certain regions such as the TPT could not be reliably reconstructed in all of the subjects. Anatomy featuring intricate but complex cross-sections, such as the thalamoinsular projections, which involves the claustrum, and medullary decussations of the spinal trigeminal fibers will remain out of reach of the current algorithms without substantial improvement in DWI resolution. Fortunately, SAGIT can adopt MR images of any resolution, so future studies in higher isovoxel resolution can be immediately processed as soon as they become available. 

The results of the thesis ultimately depend on the algorithm performances at each step. The SAGIT framework relies on the accurate image registrations to perform template, and tractography deformations. Therefore any inaccuracies in the multi-step registration pipeline will become cumulative. We went to great length to determine the best method in order to minimize registration error on a massive scale (See \ref{section:study2}), and we stress the importance to strive for registration accuracy. 



\section{Future Directions}

The methods developed in this thesis, including SAGIT and merged-tractography GP classifiers,  are not limited to the application of TN. The methods can generalize to any white matter anatomy where consistent tracts can be obtained at a group level. The limitation of its examination power is often in the data, both in data volume, resolution, and quality.

Extensions to the current studies can follow three lines of investigations:

I) Keep pushing the limit of anatomical delineations towards other areas of the trigeminal system. This include the spinal trigeminal pathways, which include the spinal trigeminal ganglion and its sub-ganglions, and its medullary decussations. Re-examine TPT tracts, both to establish its diffusivity baseline, and to confirm its asymmetry. Image the medial pain pathways, and descending projections to the PAG. These tasks are currently limited by the resolution of availabile DWI images. Immediate gains may be obtained by using SAGIT to delineate these structure on the Human Connectome Project (HCP) dataset. Additionally, investigation into 7T DWI images will be necessary. 

II) Re-examine TN and other pain in the MS population with S1/S2 and TPT pathways. The localization ability of merged-tractography analysis can be readily applied to white-matter lesion studies, and may reveal additional insights in the MS population. Where focal lesion groups can be correlated with disease symptoms such as pain and motor disabilities. Comparitive studies with other trigeminal pain syndromes such as Herpetic neuralgia and trigeminal deafferentation pain would also be a valuable direction.

III) Continue to advance machine learning to enable disease diagnosis and prognosis from white-matter tractography. Investigate how to diagnose and provide treatment targets from MR images with minimal turn around time. This can involve apply GP regression to the prediction of patient disease, and multi-modal learning, possibly with deep neural networks. Gaussian Process can be used as a regressor to model tissue parameters along merged white-matter bundles. Any MR scalar image can be embedded into the tractography streamlines, so tissue quantifications such as G-ratio, myelin, proton-density, kurtosis, magnetization transer, PET images can be examined. Since GP is a Bayesian method, the derived GP model can be used to provide predictive confidence on centroid measures from single subjects. When combined with the end-to-end tractography extraction ability of SAGIT, this provides a basis to build a deployable framework for disease inference on new patients. 


We observed that merged tractography can delineate more complete pathway delineations than from any single subject. This is similar to the observation that averaged MRI anatomical images from multiple subjects can produce high quality templates. Therefore merged tractography may serve to produce high quality tractography templates, that can form the basis for both geometric and measurement comparisons in individuals. We have explored the use of SAGIT in generating the first neonatal porcine tractography template \cite{Zhong2016a}. Such template does not exist for humans to the best of our knowledge, and can be readily created from Human Connectome Project data. In patients where tractography cannot be reliably delineated, for example in the case of severe white matter degeneration in MS or low resolution clinical acquisitions, these template tracts may provide a standard population distribution for GP regression, and can used as proxy to enable advanced analysis.

The pain neuromatrix, as studied by various functional and structural connectivity studies, distributes over the full range of neocortical areas. These areas are associated with complex white-matter pathways that extends to all regions of the brain. These include sharp curvatures of the anterior pain pathways of the ACC, the distant projections of the temporal cortices. Some pathways are just beyond the feasibility of current methods, these include the projections of the insula and claustrum, small decending pathways to PAG, and the spinal trigeminal pathways in the brainstem. Detailed examination of the pain matrix, such as the medial pain-related white matter projections of the ACC, and posterior insula would be of great interest. Multi-modal and multi-variable study with functional connectivity with this technique would also be possible. 

Tractography, in conjuction with 3D anatomical segmentation, can 1) reconstruct white matter pathways that are otherwise invisible, and 2) allow intuitive spatial relationship of key anatomies that are hard to visualise in 2D. We had previously demonstrated these capabilities in the reconstruction of deformed facial/vestibular nerve complex in relation to vestibular schwannomas \cite{Chen2011b,Behan2017}.
Another contribution is to visualize tissue changes from surgical interventions that can otherwise be invisible. There are cases where post-operative GammaKnife ablation target on the trigeminal nerve did not result in dicernable gadonenium enhancement in standard MR, but the nerve FA changes can be seen in the delineated tractography. Therefore tractography can provide both spatial, as well as tissue-based insights to the surgeons to aid in assessment.
Tractography and 3D anatomical model ultimately are standard 3D geometries in computer graphics. Therefore advances in 3D printing can be easily adopted in neurosurgery to allow surgeons to physically examine tractography reconstructions. On the otherhand, technologies such as virtual reality (VR) and augmented reality (AR), which are simply different ways to present computer graphics, can allow new ways to empower surgeons in decision making. Tractography can play a key role in bridging computer vision based diagnosis and surgical disease intervention to improve surgical outcomes and improved quality-of-life for patients.    

\section{Conclusions}

In sum, this thesis provides novel evidence on the diffusivity tissue disruptions of the CN V in TN, and that there is evidence of bilateral CN V tissue changes in TN patients when compared to controls. The examination was also extended to the thalamocortical S1 pathway and have demonstrated localized white matter changes near the white-grey matter boundary. These examinations on the group level were made possible by the development of the SAGIT merged tractography framework, that can be generally applied to group-based white matter studies as a whole. These results demonstrate that TN involves wide-spread changes at all levels of CNS white matter, and that the methods developed in this thesis may lead to improvements in white matter disease detection and diagnosis.