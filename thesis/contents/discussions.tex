\graphicspath{{images/discussions/}}

\chapter{General Discussions}
\section{Tractography}
\subsection{Delineating the trigeminal nerve into the brainstem}
The primary motivation to visualize the trigeminal nerve via tractography, was due to the inability to discern both the brainstem trigeminal nerve, nor the trigeminal nucleus in all known MRI modalities. Before tractography, the curvature of the trigeminal nerve could only be inferred from previous anatomical studies, or by postmortem dissection. 
For successful tractography of the CN V, both DWI sequence and the algorithm choice is paramount. Given a DWI sequence, at 25 gradient directions, cisternal CN V could be readily reconstructed with Gaussian tensors. Resolution of the DWI sequence would affect the output. Unlike major white matter bundles, cranial nerves diameters are in the milimeter range, and therefore voxel sizes greater than 2 mm will result in severe partial volume of cranial nerves. Achieving 2mm isovoxel on a 3T scanner requires 32 channel head coil. Standard clinical 3T MRI scanners with 8 channels head coils can only achieve 2.6 mm isovoxel resolution before acquisition time and noise becomes insurmountable. For this reason, the acquisition in-plane resolution can be tune to 1x1 mm, and slice thickness at 3mm. While the acquisition is not ideal, it sacrifices resolution in only the z-axis, the tractography visual fidelity is improved. 
Discerning brainstem CN V requires algorithms that can reliably resolve tract structures in regions with crossing fibers. A number of methods have been devised, including probabilistic tractography made popular by the FSL software package, which generates image-based results that represent the propagation probability of a seed region. Neurosurgeons however often need visualizations that represent anatomy geometrically. Therefore streamline-based tractography visualizations are of broader appeal. In our experience, both XST and sphere harmonic based propagation algorithms can perform adequately to delineate brainstem CN V, as long as there are greater then 50 gradient directions. 
Another often overlooked area is the seeding pattern. When tracing the streamlines, an algorithm will need to generate a number of starting points in a defined volume. Common patterns in prepackaged software include in a regularly spaced grid, where seed spacing can make decisive differences in visuals, and random sampling. The seeding pattern is important in both the final visual style of the tracts, as well as if key anatomical landmarks can be properly tracked. Another consideration is the reproducibility of the desired anatomy, as randomly seeded bundles cannot be reproduced exactly. Grid seeding also can produce sparse spacing of the tract bundles when viewed from a certain angle, while random seeding can reduce this kind of visual artifact. A good balance might be to have a reproducible sampling scheme that also is robust to different viewing directions, for example a phyllotaxis pattern.

\subsection{Tractography group registration}
Tractography is a powerful methodology that reveals the morphology of brain white matters that were otherwise previously invisible. Its primary usage however has been limited to case studies and small-scale anatomy comparative analysis. The continuous advancements in big data and machine learning demand the adaptation of new neuroimaging methods to the analysis and quantification of large groups.

Group comparative studies require the establishment of anatomical consensus, such that similar white matter structures between populations can be readily studied. Efforts in MRI medical image registration have paved the way to this end, especially in the fields of diffeomorphic registration. The difficulty in adapting registration to tractography is that tractography data are 3D geometries, and therefore are extremely difficult for analytical methods. The best efforts to date apply affine transformation for between-tract registrations \cite{Garyfallidis2015}. However non-linear geometric registration is still a topic of future research. Our approach is to recast tractography as a brain segmentation method that is spatially-aware. Tractography then can be considered a point-based sub-sampling method that is represented in the same coordinate frame as the native DWI space. Any transforms applied to this DWI space is then applicable also to any tractography models that share its coordinate frame. Existing diffeomorphic registration solutions can then be used to non-linearly distort tractography, and circumvent the computational difficulty of geometry based non-linear registration. The advantage of our tractography registration strategy is that any new advances in the field of MR registration can be readily adapted to improve the output of our method. 

Group registration of tractography towards a common template space requires two separate steps: T1---DWI co-registration, and T1---Template registration. Any registration errors in these two steps will become multiplicative in the end result. T1 registration is well studied, however T1---DWI registration has few literature. Therefore it was imperative that we investigated the best method to minimize registration errors that were compatible with our image data format.

\subsection{Measurements from tracts}
Aggregating along-the-measure statistics is a form of clustering that requires projection of tract vertices onto a structural skeleton, or centroid. Common clustering methods calculate a polyline centroid based on different distancing functions between streamlines. These methods,  however, require tractography streamlines to have similar length, and will underperform when tractography structures contain sharp turns. We have explored a number of strategies, and considered adaptation of manifold-based methods, and point-cloud skeletal extraction method from computer graphics literature. These approaches, however, contain assumptions that do not suit our needs. For example, manifolds commonly consider that the latent structure of the data lies on a lower-dimensional surface, while tractography is commonly arranged in complex bundles. Moreover, point-cloud methods deal with point-cloud data obtained from 3D scanners, which situate on a 3D shell, while tractography points are distributed within a volume. White matter anatomy can have intricate crossings, multi-way join and diverging structures. These are difficult for existing 3D analytical methods, and require models with simplified assumptions to reduce the number of unknown parameters. A potentially promising future direction may be to automatically discover these more complex latent structures using deep learning methods. 

\section{Trigeminal Neuralgia}

The findings in Study I \ref{section:study1} revealed localized CN V tissue differences in MS-TN and idiopathic TN. The possible implication is that disruption anywhere along the primary trigeminal afferent may lead to similar TN-like pain syndromes. MS is a classic myelin inflammatory disease, and therefore these diseases may share similar disruptions in myelin ultrastructure.

Study IV further suggested that the unaffected CN V diffusivity is also different from that of controls. The unaffected side was previously presumed to be similar to normal tissue due to the absence of pain. In many studies it has also been used as a basis of comparison. This finding however is not as surprising as one may suspect. It has been long observed that bilateral neurovascular compression could be observed in some unilateral TN cases. Moreover, in third-degree neurovascular compression, severe vascular hardening and deformation can be observed in the cisternal space. It is reasonable that these observable macroscopic changes can effect CNS tissue changes on the unaffected side. Moreover descending modulation from PAG can become maladaptive or ineffective in chronic pain, and in these cases more centrally involved changes in the trigeminal nucleus and its associated pathways are possible.

In Gamma-Knife radiosurgery treatment for TN, the target is on the cistern CN V segment, and situates proximally to the CN V REZ. Paradoxically, the GK treatment can be seen as a willful disruption of the CN V myelination. The question is, why would additional GK lesions on the CN V myelin effectively decrease TN pain. Ephaptic transmission of action potential from the fast conducting fibers to the slow conducting pain fibers is thought to explain TN pain. The paroxysmal nature of TN pain, and the unreliable association between compression and symptom, however, suggest additional factors are involved. Consider that blood circulation and motion is constant within the cistern space, the trigeminal-vascular complex is a dynamically undulating structure. The force exerted onto the CN V from its neighbouring vasculature is therefore periodic. It is possible that during the period where the vascular pressure onto the CN V is at peak, and at the same time the action potential conductance is near the locus of compression, that the ephaptic transmission is triggered and amplified, creating a "resonance condition". This would imply that given periodic vascular forces, and a constant stream of innocuous afferents, the chance of resonance condition is small, and can result in no observable pattern or interval, thus explaining the paroxysmal nature of TN pain. Moreover, under bilateral vascular compression, only the side with vascular and CN V physiology that can result in elevated chance of such resonance condition will result in pain. This model also explains that GK intervention disrupts the chance of resonance condition, thereby reducing TN pain. The pain eventually can return however, as the peripheral lesion site heals and the resonance condition returns. MVD, in this case can be explained as the physical disruption of the condition.

\section{Limitations}

One limitation across the studies in this thesis was the choice of anisotropic voxels of the DWI sequences. This was limited by the clinical 3-channel GE 3T HDx MR scanner, which places constraints both the acquisition resolution and time. Commonly, DWI voxel resolution on a 32-channel 3T research scanner can obtain up to 2mm isovoxel resolution, on older 8-channel clinical scanners, the maximum isovoxel resolution is limited at about 2.6mm isovoxel. This is insufficient for the study of cranial nerves such as CN V. Our resolution is limited to 1x1x3 mm voxel dimensions, in order to retain in-plane resolution at the sacrifice of slice thickness. Despite this, certain regions such as the TPT could not be reliably reconstructed in all of the subjects. Anatomy featuring intricate but complex cross-sections, such as the thalamoinsular projections, which involves the claustrum, and medullary decussations of the spinal trigeminal fibers will remain out of reach of the current algorithms without substantial improvement in DWI resolution. Fortunately, SAGIT can adopt MR images of any resolution, so future studies in higher isovoxel resolution can be immediately processed as soon as they become available. 

The results of the thesis ultimately depend on the algorithm performances at each step. The SAGIT framework relies on the accurate image registrations to perform template, and tractography deformations. Therefore any inaccuracies in the multi-step registration pipeline will become cumulative. We went to great length to determine the best method in order to minimize registration error on a massive scale (See \ref{section:study2}), and we stress the importance to strive for registration accuracy. 



\section{Future development}

The methods developed in this thesis, including SAGIT and merged-tractography GP classifiers,  are not limited to the application of TN. The methods can generalize to any anatomy where consistent tracts can be obtained at a group level. The limitation of its examination power is often in the data, both in data volume, resolution, and quality. 

An immediate application is to re-examine TPT tracts, both to establish its diffusivity baseline, and to confirm its asymmetry. This can be immediately deployed on the Human Connectome Project (HCP) dataset. Additional examine in a diseased population requires the availability of clinical MR images that at minimum matches the parameters of HCP. 

For pain studies, detailed examination of the pain matrix, such as the medial pain-related white matter projections of the ACC, and posterior insula would be of great interest. Multi-modal and multi-variable study with functional connectivity with this technique would also be possible. 

The localization ability of merged-tractography analysis can be readily applied to white-matter lesion studies, and may reveal additional insights in the MS population. Where focal lesion groups can be correlated with disease symptoms such as pain and motor disabilities. 

Gaussian Process can be used as a regressor to model tissue parameters along merged white-matter bundles. Any MR scalar image can be embedded into the tractography streamlines, so tissue quantifications such as G-ratio, myelin, proton-density, kurtosis, magnetization transer, PET images can be examined. Since GP is a Bayesian method, the derived GP model can be used to provide predictive confidence on centroid measures from single subjects. When combined with the end-to-end tractography extraction ability of SAGIT, this provides a basis to build a deployable framework for disease inference on new patients. 

\section{Conclusions}

In sum, this thesis provides novel evidence on the diffusivity tissue disruptions of the CN V in TN, and that there is evidence of bilateral CN V tissue changes in TN patients when compared to controls. The examination was also extended to the thalamocortical S1 pathway and have demonstrated localized white matter changes near the white-grey matter boundary. These examinations on the group level were made possible by the development of the SAGIT merged tractography framework, that can be generally applied to group-based white matter studies as a whole. These results demonstrate that TN involves wide-spread changes at all levels of CNS white matter, and that the methods developed in this thesis may lead to improvements in white matter disease detection and diagnosis. 



 

