\graphicspath{{images/discussions/}}

\chapter{General Discussions}

\section{Summary Of Findings}

The contribution of this thesis are as follows:

\begin{itemize}

\item We are the first to image the brainstem segment of the CN V \textit{in vivo} using multi-tensor diffusion tractography. Prior to this work, CN V tractography was limited to single-tensor DTI models and cannot be delineated into the pons. Clear delineation of the pontine CN V nerve allows reliable measurements of CN V diffusivity, therefore greatly increase the range of trigeminal-related pathology, for example MS-TN, that can be studied with neuroimaging. 

\item The study, for the first time, clearly differentiated TN and MS-TN by sub-segment diffusivity differences, between cistern (TN) and brainstem peri-lesional (MS-TN) segments. This distinction in the trigeminal REZ between these two conditions have been suspected, but never confirmed. The study is the first to suggest that disruptions along the CN V in different locations may trigger similar pain symptoms.

\item We overcame the major technical barrier to detailed diffusion tractography study of the trigeminal pathway in a large population group, by developing and validate the Selective Automated Group Integrated Tractography (SAGIT) software platform. In the process, we determined the optimal T1-DWI co-registration methodology with clinical DWI sequences. Prior to this project, the co-registration method has never been exhaustively determined and reported in literature.

\item We are the first to perform merged group tractography study of the trigeminal nerve, and its primary sensory pathway, and the first to conduct such study in TN, and the first to perform Gaussian Process machine learning to the study of TN. We studied the trigeminal primary (CN V), secondary (trigeminopontothalamic, TPT), and tertiary (trigeminocortical, S1) white patter pathways in its cross-sectional detail. We discovered that the unaffected CN V diffusivity differs from controls as well as the affected side and that the affected TPT pathway and S1 pathway diffusivity differs from controls.

\end{itemize}

\section{Trigeminal Neuralgia}

The findings in Study I (Chapter \ref{section:study1}), which aimed to identify trigeminal nerve diffusivity signatures in TN and MS-TN, identified for the first time a method that can distinguish between microstructural changes in these conditions, and revealed localized CN V diffusivity differences in MS-TN and idiopathic TN. The possible implication is that disruption anywhere along both the cisternal and pontine segments of the trigeminal nerve may lead to similar TN-like pain syndromes. Central demyelination in the trigeminal root transition zone is associated with vascular compression in TN. Moreover MS is a classic myelin inflammatory disease. Therefore both diseases may share pathophysiology due to disruptions in myelin ultrastructure. The study opened the possibility that multi-tensor tractograpy aided analysis added a measure of objective assessment in examining pain conditions, and may facilitate similar studies that provide insight into the fundamental changes that occur at the level of white matter. 

Study IV (Chapter \ref{section:study4}), which applied group tractography to the study of the trigeminal sensory pathway, further suggested that the unaffected CN V diffusivity is also different from that of controls. DeSouza et al. \cite{Desouza2013} also presented similar findings, where the unaffected RD, MD, and AD were found to differ from controls. Desouza performed the study with manual ROI placement at a single measurement position in the REZ, Whereas we performed GP classification along the nerve automatically. The unaffected side was previously presumed to be similar to normal tissue due to the absence of pain. In many studies it has also been used as a basis of comparison. This finding, however, is not surprising. It has been long observed that bilateral neurovascular compression could be observed in some unilateral TN cases. Moreover, in third-degree neurovascular compression, severe vascular hardening and deformation can be observed in the cisternal space. It is reasonable that these observable macroscopic changes can effect CNS tissue changes on the unaffected side. Moreover descending modulation from PAG can become maladaptive or ineffective in chronic pain, and in these cases more centrally involved changes in the trigeminal nucleus and its associated pathways are possible. With this additional insight, it becomes important to not interpret results of affected-to-unaffected comparisons in terms of differences between affected and healthy controls. Additionally, the logical next question is to determine that, if the unaffected nerve differs from controls, at what degree of difference does it result in pain. Investigations into bilateral TN pain, and vascular compression in the absence of pain, may reveal additional insight into the relationship between white matter changes and TN pain in general.

In Gamma-Knife radiosurgery treatment for TN, the target is on the cistern CN V segment, and situates proximally to the CN V REZ. Paradoxically, the GK treatment can be seen as a willful disruption of the CN V myelination. The question is, why would additional GK lesions on the CN V myelin effectively decrease TN pain. Ephaptic transmission of action potential from the fast conducting fibers to the slow conducting pain fibers is thought to explain TN pain. The paroxysmal nature of TN pain, and the unreliable association between compression and symptom, however, suggest additional factors are involved. Consider that blood circulation and motion is constant within the cistern space, the trigeminal-vascular complex is a dynamically undulating structure. The force exerted onto the CN V from its neighbouring vasculature is therefore periodic. It is possible that during the period where the vascular pressure onto the CN V is at peak, and at the same time the action potential conductance is near the locus of compression, that the ephaptic transmission is triggered and amplified, creating a "resonance condition". This would imply that given periodic vascular forces, and a constant stream of innocuous afferents, the chance of resonance condition is small, and can result in no observable pattern or interval, thus explaining the paroxysmal nature of TN pain. Moreover, under bilateral vascular compression, only the side with vascular and CN V physiology that can result in elevated chance of such resonance condition will result in pain. This model also explains that GK intervention disrupts the chance of resonance condition, thereby reducing TN pain. The pain eventually can return however, as the peripheral lesion site heals and the resonance condition returns. MVD, in this case can be explained as the physical disruption of the condition.

\section{Tractography of the trigeminal nerve and brainstem fibers}

The primary motivation for the study of the trigeminal nerve with tractography was the significant limitation of current conventional MR imaging in the visualization of the brainstem trigeminal nerve and the area of the trigeminal nucleus. Prior to the use of tractography, the structure of the trigeminal nerve could only be inferred from direct anatomical studies or postmortem dissection. 

Resolving cross fibers of the cerebral white matter was a fundamental advancement in diffusion tractography. The brain white matter was believed to consists of 90\% complex crossing pathways, and the inability for DTI to reliably image these structure posed a fundamental challenge for neuroimaging. 
A widely adopted technique was probabilistic tractography, which used tissue priors to sample a large number of directions at each step stochastically, and combined the final result as a volumetric probability map. This kind of image-based probabilistic tractography was useful for whole-brain white matter studies. However, it was constrained to the resolution of the diffusion acquisition, which was insufficient for localized and small diameter pathways. Moreover, the algorithm depended on pre-generated tissue-priors, which excluded cranial nerves. 
Deterministic algorithms such as diffusion spectral imaging (DSI), and the related multi-shell acquisition strategies attempted to model the fiber orientation distribution by dramatically increasing the number of gradient directions (256--512 directions were required) in the diffusion acquisition. The acquisition time of these sequences was impractical for the clinical environment. Reduction in acquisition time for these algorithms is currently an active area of research \cite{}.
The family of spherical harmonics (SH) based methods, which include Q-ball and constrained spherical deconvolution (CSD), were a significant development. SH-based methods modelled the fiber orientation distribution as a decomposition of SH basis functions. This meant that the methods are more robust to noise, and therefore require less number of gradient acquisitions. CSD proved to be particularly robust to noise, and can be deployed on acquisitions with only 55 directions. Another family of tractography (XST) algorithm constrained its model assumption to two fiber crossings, also proved to provide good results with 50 gradient directions.

These key algorithmic developments decidedly brought cross-fiber tractography in clinical neuroimaging to the realm of possibility. This is significant as the vast majority of imaging of neuropathologies in patients only exist as clinical acquisitions. We took advantage of this opportunity and explored the feasibility of imaging the pontine trigeminal nerve with CSD and XST. The trigeminal nerve is an ideal imaging target, both due to our immediate interest in TN, and also that the trigeminal nerve has a number of unique properties as a tractography target: 1) delineation quality is easy to judge due to its simple morphology. 2) it situates in both single-fiber (cistern) and crossing-fiber environments; 3) it is the largest cranial nerve, and therefore a good benchmark for imaging small anatomy.

The ability to resolve the pontine trigeminal nerve into the trigeminal nucleus was encouraging. It signified that for the first time we could segment the pontine trigeminal anatomy reliably with computer aid. Previously its identification was done by eye, and sometimes with simple heuristics based on distance from surrounding landmarks. Such manual methods were highly subjective and became problematic with patient images, where distortions to the nerve could not be assessed. 

The ability to segment the entire trigeminal nerve also opens the avenue to measure its tissue diffusivity from tractography directly. Moreover, specific tissue-metrics such as myelin image, or PET images, can also be measured with confidence in reproducibility. This extends these methods to facilitate disease studies well beyond TN. We were able to divide the trigeminal nerve into 30 discrete cross-sections along its path, and observed as never before, detailed diffusivity changes as the nerve transits from pontine tissue to the cistern. Determining a common centroid curve from many subjects was computationally non-trivial, especially when the streamline lengths were highly variable, due to the noise in the cistern CSF. Turns out that the merged tractography strategy simplified this task by allowed a single manual curve to be defined manually in the template tractography space. This strategy can similarly be extended for obtaining along-the-track measure for other highly variable streamline bundles.

We applied machine learning to the analysis of TN in Study IV in order to avoid our own confirmation bias. The analysis of the trigeminal nerve in TN versus controls served as validation for the study's method. We had performed traditional multiple comparison statistical analysis, but found the result hard to interpret without inducing biases. The GP classifier was chosen, over more popular SVM classifiers, for the following reasons: 1) the obtained metrics show high location dependency. 2) SVM results are hard to interpret, whereas GP is a bayesian method, and is much more intuitive. 

\section{Neuroimaging of deeper white matter for the study of pain}

The pain neuromatrix, as studied by various functional and structural connectivity studies, distributes over the full range of neocortical areas. These areas are associated with complex white-matter pathways that extends to all regions of the brain. These include sharp curvatures of the anterior pain pathways of the ACC, the distant projections of the temporal cortices. Some pathways are just beyond the feasibility of current methods, these include the projections of the insula and claustrum, small decending pathways to PAG, and the spinal trigeminal pathways in the brainstem.

We have shown in Study III (\ref{section:study3}), for the first time, that different tractography algorithms are suited to different neuropathways when comparing merged reconstructions in 42 subjects. Maier-Hein et al came to similar conclusions after examining 96 reconstructions from a single set of simulated data \cite{Maier-Hein2017}.
Therefore care must be taken to ensure tractography parameters are targeted to the anatomy of study, as more consistent delineation lead to more robust tissue measurements.

We observed that merged tractography can delineate more complete pathway delineations than from any single subject. This is similar to the observation that averaged MRI anatomical images from multiple subjects can produce high quality templates. Therefore merged tractography may serve to produce high quality tractography templates, that can form the basis for both geometric and measurement comparisons in individuals. We have explored the use of SAGIT in generating the first neonatal porcine tractography template \cite{Zhong2016a}. Such template does not exist for humans to the best of our knowledge, and can be readily created from Human Connectome Project data. In patients where tractography cannot be reliably delineated, for example in the case of severe white matter degeneration in MS or low resolution clinical acquisitions, these template tracts may be used as proxy to enable analysis.


\section{Neurosurgery and Tractography}

The biggest value of tractography to neurosugery is to assit in pre-operative planning. Tractography, in conjuction with 3D anatomical segmentation, can 1) reconstruct white matter pathways that are otherwise invisible, and 2) allow intuitive spatial relationship of key anatomies that are hard to visualise in 2D. We had previously demonstrated these capabilities in the reconstruction of deformed facial/vestibular nerve complex in relation to vestibular schwannomas \cite{Chen2011b,Behan2017}. Tractography and 3D anatomical model ultimately are standard 3D geometries in computer graphics. Therefore advances in 3D printing can be easily adopted in neurosurgery to allow surgeons to physically examine tractography reconstructions. On the otherhand, technologies such as virtual reality (VR) and augmented reality (AR), which are simply different ways to present computer graphics, can allow new ways to empower surgeons in decision making. Tractography can play a key role in bridging computer vision based diagnosis and surgical disease intervention to improve surgical outcomes and improved quality-of-life for patients.    


\section{Method Discussions}

\subsection{Delineating the brainstem fibers of the trigeminal nerve}

For successful tractography of the CN V, both DWI sequence and the algorithm choice are paramount. Given a DWI sequence, cisternal CN V could be readily reconstructed with Gaussian tensors. Unlike major white matter bundles, cranial nerves diameters are in the millimetre range, and therefore voxel sizes greater than 2 mm may result in partial volume averaging. Achieving 2mm isovoxel on a 3T scanner requires 32 channel head coil. Standard clinical 3T MRI scanners with 8 channels head coils can only achieve 2.6 mm isovoxel resolution before acquisition time and noise becomes insurmountable. For this reason, the acquisition in-plane resolution can be tuned to 1x1 mm, and slice thickness at 3mm. While the acquisition is not ideal, it sacrifices resolution only in the z-axis, however the tractography visual fidelity is improved. 
Imaging the brainstem CN V requires algorithms that can reliably resolve tract structures in regions with crossing fibers. A number of methods have been devised, including probabilistic tractography (FSL)\cite{Behrens2007}, which generates image-based results that represent the propagation probability of a seed region. Neurosurgical visualization however requires three dimensional representation of the fibers, and therefore streamline-based tractography visualizations are of broader appeal. In our experience, both XST and sphere harmonic based propagation algorithms can perform adequately to delineate brainstem CN V, as long as there are greater then 50 gradient directions. 

An important technical point, but rarely mentioned in literally, is the tractography seeding pattern. Therefore I will address it here: When tracing the streamlines, an algorithm will need to generate a number of starting points in a defined volume. Common patterns in prepackaged software include in a regularly spaced grid and random sampling. The seeding pattern is important in both the final visual representation of the tracts, and to determine whether key anatomical landmarks can be properly tracked. Another consideration is the reproducibility of the desired anatomy, as randomly seeded bundles cannot be reproduced deterministically. Grid seeding may result in sparse spacing of the tract bundles when viewed from a certain angle, while random seeding can reduce this kind of visual artefact. A good balance might be to have a reproducible sampling scheme that also is robust to different viewing directions, for example a phyllotaxis pattern.

\subsection{Tractography group registration}
Tractography is a powerful methodology that permits the visualization of white matter fibers in three dimensions. Its primary use however has been limited to case studies and small-scale anatomy comparative analysis. The continuous advancements in big data and machine learning demand the adaptation of new neuroimaging methods to the analysis and quantification of larger groups.

Group comparative studies require the establishment of anatomical consensus, such that similar white matter structures between populations can be readily studied. Efforts in MRI medical image registration have paved the way to this end, especially in the fields of diffeomorphic registration. The difficulty in adapting registration to tractography is that tractography data are 3D geometries, and therefore are extremely difficult for analytical methods, as they are computationally untenable. To date there is no direct non-linear 3D deformation methods for tractography. The best efforts to date apply affine transformation for between-tract registrations \cite{Garyfallidis2015}. Our approach is to recast tractography as a brain segmentation method while also make use of its orientation information. Tractography then can be considered a point-based sub-sampling method that is represented in the same coordinate frame as the native DWI space. Any transforms applied to this DWI space is then applicable also to any tractography models that share its coordinate frame. Existing diffeomorphic registration solutions can then be used to non-linearly distort tractography, and circumvent the computational difficulty of geometry based non-linear registration. The advantage of our tractography registration strategy is that any new advances in the field of MR registration can be readily adapted to improve the output of our method. 

Group registration of tractography towards a common template space requires two separate steps: T1---DWI co-registration, and T1---Template registration. Any registration errors in these two steps will become multiplicative in the end result. T1 registration is well studied, however T1---DWI registration has few literature. This demonstrates the need to investigate the best method to minimize registration errors that were compatible with our image data format.

\subsection{Measurements from tracts}
Aggregating along-the-measure statistics is a form of clustering that requires projection of tract vertices onto a structural skeleton, or centroid. Common clustering methods calculate a polyline centroid based on different distancing functions between streamlines. These methods require tractography streamlines to have similar length, and will underperform when tractography structures contain sharp turns. We have explored a number of strategies, and considered adaptation of manifold-based methods, and point-cloud skeletal extraction method from computer graphics literature. These approaches, however, contain assumptions that do not suit our needs. For example, manifolds commonly consider that the latent structure of the data lies on a lower-dimensional surface, while tractography is commonly arranged in complex bundles. Moreover, point-cloud methods deal with point-cloud data obtained from 3D scanners, which situate on a 3D shell, while tractography points are distributed within a volume. White matter anatomy can have intricate crossings, multi-way join and diverging structures. These are difficult for existing 3D analytical methods, and require models with simplified assumptions to reduce the number of unknown parameters. A potentially promising future direction may be to automatically discover these more complex latent structures using deep learning methods.

\section{Limitations}

One limitation across the studies in this thesis was the choice of anisotropic voxels of the DWI sequences. This was limited by the clinical 3-channel GE 3T HDx MR scanner, which places constraints both the acquisition resolution and time. Commonly, DWI voxel resolution on a 32-channel 3T research scanner can obtain up to 2mm isovoxel resolution, on older 8-channel clinical scanners, the maximum isovoxel resolution is limited at about 2.6mm isovoxel. This is insufficient for the study of cranial nerves such as CN V. Our resolution is limited to 1x1x3 mm voxel dimensions, in order to retain in-plane resolution at the sacrifice of slice thickness. Despite this, certain regions such as the TPT could not be reliably reconstructed in all of the subjects. Anatomy featuring intricate but complex cross-sections, such as the thalamoinsular projections, which involves the claustrum, and medullary decussations of the spinal trigeminal fibers will remain out of reach of the current algorithms without substantial improvement in DWI resolution. Fortunately, SAGIT can adopt MR images of any resolution, so future studies in higher isovoxel resolution can be immediately processed as soon as they become available. 

The results of the thesis ultimately depend on the algorithm performances at each step. The SAGIT framework relies on the accurate image registrations to perform template, and tractography deformations. Therefore any inaccuracies in the multi-step registration pipeline will become cumulative. We went to great length to determine the best method in order to minimize registration error on a massive scale (See \ref{section:study2}), and we stress the importance to strive for registration accuracy. 



\section{Future development}

The methods developed in this thesis, including SAGIT and merged-tractography GP classifiers,  are not limited to the application of TN. The methods can generalize to any anatomy where consistent tracts can be obtained at a group level. The limitation of its examination power is often in the data, both in data volume, resolution, and quality. 

An immediate application is to re-examine TPT tracts, both to establish its diffusivity baseline, and to confirm its asymmetry. This can be immediately deployed on the Human Connectome Project (HCP) dataset. Additional examine in a diseased population requires the availability of clinical MR images that at minimum matches the parameters of HCP. 

For pain studies, detailed examination of the pain matrix, such as the medial pain-related white matter projections of the ACC, and posterior insula would be of great interest. Multi-modal and multi-variable study with functional connectivity with this technique would also be possible. 

The localization ability of merged-tractography analysis can be readily applied to white-matter lesion studies, and may reveal additional insights in the MS population. Where focal lesion groups can be correlated with disease symptoms such as pain and motor disabilities. 

Gaussian Process can be used as a regressor to model tissue parameters along merged white-matter bundles. Any MR scalar image can be embedded into the tractography streamlines, so tissue quantifications such as G-ratio, myelin, proton-density, kurtosis, magnetization transer, PET images can be examined. Since GP is a Bayesian method, the derived GP model can be used to provide predictive confidence on centroid measures from single subjects. When combined with the end-to-end tractography extraction ability of SAGIT, this provides a basis to build a deployable framework for disease inference on new patients. 

\section{Conclusions}

In sum, this thesis provides novel evidence on the diffusivity tissue disruptions of the CN V in TN, and that there is evidence of bilateral CN V tissue changes in TN patients when compared to controls. The examination was also extended to the thalamocortical S1 pathway and have demonstrated localized white matter changes near the white-grey matter boundary. These examinations on the group level were made possible by the development of the SAGIT merged tractography framework, that can be generally applied to group-based white matter studies as a whole. These results demonstrate that TN involves wide-spread changes at all levels of CNS white matter, and that the methods developed in this thesis may lead to improvements in white matter disease detection and diagnosis. 
